\documentclass{article}

\usepackage{array}
\usepackage{etoolbox}
\usepackage{fancyhdr}
\usepackage{geometry} 
\usepackage{graphicx}
\usepackage{soul}
\usepackage{titling}

%%%%%%%%%%%%%%%%%%%%%%%%%%%%%%%%%%%%%%%%%%%%%%%%%%%%%%%%%%%%
% BEGIN METADATA: Edit the following as appropriate
%%%%%%%%%%%%%%%%%%%%%%%%%%%%%%%%%%%%%%%%%%%%%%%%%%%%%%%%%%%%

\title{Project Title}  % the title of your project
\newcommand\shorttitle{\thetitle}  % if needed: a shorter title for the document header
% Team members.
\newcommand\firstname{Ifrah Chishti}  % full name
\newcommand\firstid{ic08351}         % ID, e.g. xy01234
\newcommand\secondname{Naaseh Sajid} % full name
\newcommand\secondid{ms08085}        % ID, e.g. xy01234
\newcommand\thirdname{Aina Shakeel}  % full name
\newcommand\thirdid{as08430}         % ID, e.g. xy01234
% Uncomment the rows for the next 2 students if and as needed.
\newcommand\fourthname{Zain Hatim} % full name
\newcommand\fourthid{zh08343}        % ID, e.g. xy01234
% \newcommand\fifthname{Student 5}  % full name
% \newcommand\fifthid{id05}         % ID, e.g. xy01234

%%%%%%%%%%%%%%%%%%%%%%%%%%%%%%%%%%%%%%%%%%%%%%%%%%%%%%%%%%%%
% END METADATA: Do not edit the preamble any further.
%%%%%%%%%%%%%%%%%%%%%%%%%%%%%%%%%%%%%%%%%%%%%%%%%%%%%%%%%%%%

\pagestyle{fancy}
\lhead{Kaavish Proposal}
\chead{\shorttitle}
\rhead{Fall 2025}
\cfoot{Page \thepage}
\renewcommand{\footrulewidth}{0.4pt}

\newenvironment{instruction}{\itshape}{}

\begin{document}

% Cover page.
\input{cover}

%%%%%%%%%%%%%%%%%%%%%%%%%%%%%%%%%%%%%%%%%%%%%%%%%%%%%%%%%%%%
% DATA: Populate the rest of the document as instructed.
%%%%%%%%%%%%%%%%%%%%%%%%%%%%%%%%%%%%%%%%%%%%%%%%%%%%%%%%%%%%
\section{Abstract}
\begin{instruction}
    CodeSchool is a global coding literacy platform for children aged 6 - 16, delivering more than 2,000 classes across 20 countries. With a mission to make coding a universal literacy, the organization combines an engaging, game-based curriculum with trained instructors who can inspire the next generation of digital creators. Much like how Uber seamlessly connects riders with drivers, CodeSchool envisions a platform that can intelligently connect students with the right instructor, at the right time, using the right curriculum—whether classes are delivered online or onsite. To achieve this vision at scale, the platform must evolve from a system dependent on manual operations into one that is automated, intelligent, and easy to use.

    Currently, managing the lifecycle of classes at CodeSchool is a complex, time-consuming process. Instructors and administrators rely on spreadsheets to schedule classes, assign students, generate Zoom links, and track progress. Evaluations of teaching quality are also carried out manually, with the team watching recorded sessions and giving scores based on observations. While effective at a smaller scale, this system becomes increasingly difficult to manage as CodeSchool grows, leaving room for scheduling conflicts, inefficiencies, and limited insight into teaching quality. These challenges risk slowing down the organization's ability to expand while maintaining its high standard of student learning outcomes.

    The proposed solution is a web-based platform designed to act as a central instructor portal—streamlining operations, automating repetitive tasks, and enabling data-driven insights. At its core, the platform addresses three key areas: class scheduling, student-instructor matching, and performance evaluation.

    First, the platform will automate class scheduling. By factoring in instructor availability, coding language expertise, and student age and proficiency, it will generate conflict-free schedules. This not only saves administrative effort but also ensures that classes run smoothly without last-minute rescheduling.

    Second, the platform will transform how students are assigned to instructors. Instead of a manual process that relies on human judgment, the system will use a recommendation model inspired by machine learning. Much like matching services in other industries, this model will prioritize placing students with instructors best suited to their learning needs—whether that is based on teaching style, technical strengths, or experience with specific age groups. This will also follow a round robin model to ensure fair distribution of students among instructors, allowing instructors to experience teaching different age groups and fostering a more versatile teaching approach. The result is a more personalized and effective learning experience for every student.

    % Third, to support the logistics of virtual teaching, the platform will integrate with the Zoom API to automatically generate and distribute meeting links for each scheduled class. This eliminates another manual step and ensures that both students and instructors always have seamless access to their sessions.

    Finally, the platform will introduce a more advanced approach to instructor performance analysis. Rather than relying solely on manual reviews, it will leverage automated transcription and analysis of Zoom recordings. Using metrics such as clarity, pacing, student engagement, and adherence to curriculum (just to name a few), the system will provide instructors with actionable feedback. This not only supports professional growth but also ensures that students consistently receive high-quality instruction.

    By bringing these features together into one unified dashboard, the project aims to reduce operational overhead while enabling scalability. Instructors will gain a clear and organized view of their teaching responsibilities, student assignments, and performance metrics. Administrators will benefit from real-time data and analytics that guide decision-making. Most importantly, students will experience smoother, better-matched, and higher-quality instruction—helping CodeSchool continue its mission of making coding a universal literacy.

    In summary, this project is not just a technical upgrade but a step toward reimagining how educational platforms can operate at scale. By blending automation with data-driven insights, it empowers instructors to focus on what they do best—teaching—while the system handles the background logistics. The result is a more efficient, reliable, and impactful platform that supports CodeSchool's global ambitions.
\end{instruction}

\section{Problem definition}
\instruction{
    Currently, managing the lifecycle of classes at CodeSchool is a complex, time-consuming process. Instructors and administrators rely on a spreadsheet based system to schedule classes, assign students, generate Zoom links, and track progress. Evaluations of teaching quality are also carried out manually, with the team watching recorded sessions for each instructor and giving scores based on observations. While effective at a smaller scale, this system becomes increasingly difficult to manage as CodeSchool grows, leaving room for scheduling conflicts, inefficiencies, and limited insight into teaching quality. These challenges risk slowing down the organization's ability to expand while maintaining its high standard of student learning outcomes.
}

\section{Social relevance}
\instruction{
    The project addresses the broader societal challenge of ensuring equitable access to quality education in an increasingly digital world. As demand for coding literacy grows globally, many children—particularly in rural areas—lack access to qualified instructors and structured learning opportunities. Current manual systems of scheduling, matching, and evaluation are not scalable, limiting the reach and effectiveness of coding education platforms. By automating these processes and introducing data-driven insights, the project not only streamlines operations but also ensures that students are consistently matched with the right instructors and receive high-quality teaching. In doing so, it supports the mission of making coding a universal literacy, reduces barriers to access, and models how education systems can expand sustainably to meet the needs of diverse learners worldwide.
}

\section{Originality/Novelty}
\instruction{
    Solving this problem creates significant value for both learners and educators. By automating class scheduling, student–instructor matching, and performance evaluation, the platform removes operational bottlenecks that traditionally slow down digital education programs. Instructors gain time and clarity to focus on teaching rather than administrative work, while students benefit from being paired with instructors who best fit their learning needs—improving engagement, consistency, and outcomes. For administrators, the centralized dashboard provides real-time visibility into scheduling, payments, and teaching quality, supporting more informed decision-making and sustainable growth.

    Compared with existing solutions, which often rely on spreadsheets, messaging apps, or generic learning management systems, this platform is purpose-built for coding education at scale. Tools like Zoom and Google Classroom are excellent for video delivery and content sharing, but they do not address the deeper challenge of intelligently matching students to instructors or evaluating teaching quality. Likewise, traditional CRMs or scheduling tools may handle logistics but lack an educational focus and offer no integration with performance analytics. The added value of this project lies in combining all these elements—matching, scheduling, conferencing, and evaluation—into a single, seamless experience. Current solutions operate in silos, but this platform brings them together into one cohesive ecosystem, making it uniquely positioned to scale and sustain quality learning worldwide.
}

\section{CS contribution}
\instruction{Describe the CS component of the project, e.g. the higher level CS courses that contribute to it.}

\section{Scope and Deliverables}
\instruction{Justify the scope of the project with respect to the size of the team and the year long duration. List the foreseeable deliverables.}

\section{Feasibility}
\instruction{List the resources, e.g. datasets, compute resources, software libraries, hardware, required for the project. Mention how you expect to access and utilize them for the project.}

\section{Team dynamics}
\instruction{Justify the suitability of the team members to the project. For example, their relevant courses, projects, internships, or research.}

\section{Tech stack}
\instruction{Write details of the tech stack you will use for this project for e.g. if you are using MERN stack, you can write MongoDB, Express, React and NodeJS etc.}

\section{References}
\instruction{List your references.}

% External advisor undertaking.
\input{external}

\end{document}

%%% Local Variables:
%%% mode: latex
%%% TeX-master: t
%%% End:
