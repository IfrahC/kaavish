\documentclass{article}

\usepackage{array}
\usepackage{etoolbox}
\usepackage{fancyhdr}
\usepackage{geometry} 
\usepackage{graphicx}
\usepackage{soul}
\usepackage{titling}
\usepackage{url}

%%%%%%%%%%%%%%%%%%%%%%%%%%%%%%%%%%%%%%%%%%%%%%%%%%%%%%%%%%%%
% BEGIN METADATA: Edit the following as appropriate
%%%%%%%%%%%%%%%%%%%%%%%%%%%%%%%%%%%%%%%%%%%%%%%%%%%%%%%%%%%%

\title{CodeSchool CMS}  % the title of your project
\newcommand\shorttitle{\thetitle}  % if needed: a shorter title for the document header
% Team members.
\newcommand\firstname{Ifrah Chishti}  % full name
\newcommand\firstid{ic08351}         % ID, e.g. xy01234
\newcommand\secondname{Naaseh Sajid} % full name
\newcommand\secondid{ms08085}        % ID, e.g. xy01234
\newcommand\thirdname{Aina Shakeel}  % full name
\newcommand\thirdid{as08430}         % ID, e.g. xy01234
% Uncomment the rows for the next 2 students if and as needed.
\newcommand\fourthname{Zain Hatim} % full name
\newcommand\fourthid{zh08343}        % ID, e.g. xy01234
% \newcommand\fifthname{Student 5}  % full name
% \newcommand\fifthid{id05}         % ID, e.g. xy01234

%%%%%%%%%%%%%%%%%%%%%%%%%%%%%%%%%%%%%%%%%%%%%%%%%%%%%%%%%%%%
% END METADATA: Do not edit the preamble any further.
%%%%%%%%%%%%%%%%%%%%%%%%%%%%%%%%%%%%%%%%%%%%%%%%%%%%%%%%%%%%

\pagestyle{fancy}
\lhead{Kaavish Proposal}
\chead{\shorttitle}
\rhead{Fall 2025}
\cfoot{Page \thepage}
\renewcommand{\footrulewidth}{0.4pt}

\newenvironment{instruction}{\itshape}{}

\begin{document}

% Cover page.
\input{cover}

%%%%%%%%%%%%%%%%%%%%%%%%%%%%%%%%%%%%%%%%%%%%%%%%%%%%%%%%%%%%
% DATA: Populate the rest of the document as instructed.
%%%%%%%%%%%%%%%%%%%%%%%%%%%%%%%%%%%%%%%%%%%%%%%%%%%%%%%%%%%%
\section{Abstract}
\begin{instruction}
    This platform is a streamlined Classroom Management System (CMS) designed to enhance digital learning and improve operational efficiency for online educational organizations. Our platform’s mission is to make classroom and content management and instructional delivery easily accessible and streamlined. Combining an engaging, modular curriculum with a dynamic, intuitive system, the platform connects instructors and students while automating key processes. Similar to how ride-sharing apps efficiently connect riders with drivers, this CMS envisions intelligently pairing students with instructors, using data-driven insights and automation to ensure the right match at the right time.\newline 
    At its center, the platform addresses the challenges of managing educational content and instructor-student interactions at scale. Currently, many online educational institutions rely on manual systems, including spreadsheets and manual scheduling, to handle class assignments, content distribution, and performance tracking. This process, while manageable at smaller scales, becomes increasingly inefficient as organizations grow, leading to scheduling conflicts, inconsistencies in learning experiences, and lack of real-time insights into instructor performance.\newline
    The proposed solution is a web-based CMS that centralizes all instructor and class-related operations into a seamless platform. This system automates routine tasks, such as class scheduling and content distribution, and introduces a performance-driven approach for instructor evaluations. \newline
    First, the platform will automate the scheduling of classes by considering factors such as instructor availability, content expertise, and student learning profiles. This ensures that classes are scheduled without conflicts and with minimal administrative overhead, allowing educational institutions to focus on quality teaching and learning. \newline
    Second, the system will utilize an intelligent matching algorithm to assign students to the most appropriate instructors. Drawing on machine learning-inspired recommendations, the platform will prioritize personalized student-instructor pairings based on teaching styles, subject expertise, past performance, among other metrics. A round-robin approach will ensure fair distribution of students among instructors, fostering best student-teacher pairing and efficient mapping of resources.\newline
    The platform will also integrate seamlessly with third-party tools, such as video conferencing APIs of Zoom, to handle logistics like meeting link generation and communication, further reducing manual tasks and enhancing the overall user experience for both instructors and students. \newline
    Finally, the platform introduces advanced metrics for evaluating instructor performance. By automating the transcription and analysis of teaching sessions, the system will provide actionable feedback to instructors, helping them to continuously improve their instructional methods. Key metrics, which include, but are not limited to, clarity, pacing, engagement, and curriculum adherence, will guide instructors’ professional growth while ensuring that students consistently receive high-quality, effective instruction. \newline
    The goal of this CMS is to reduce operational complexity, providing instructors with an organized dashboard to manage their classes, schedules, and student interactions. Administrators will benefit from real-time data and analytics to drive decision-making, while students enjoy a more efficient, engaging, and personalized learning experience. \newline
    In summary, this platform offers a robust, scalable solution to the challenges of managing educational content and instructor-student interactions. By automating workflows, leveraging intelligent matching, and providing data-driven insights, it enables educational organizations to operate more efficiently and effectively, supporting the growth and success of digital learning initiatives. \newline
\end{instruction}

\section{Problem definition}
\instruction{
    Currently, managing the lifecycle of classes at CodeSchool is a complex, time-consuming process. Instructors and administrators rely on a spreadsheet based system to schedule classes, assign students, generate Zoom links, and track progress. Evaluations of teaching quality are also carried out manually, with the team watching recorded sessions for each instructor and giving scores based on observations. While effective at a smaller scale, this system becomes increasingly difficult to manage as CodeSchool grows, leaving room for scheduling conflicts, inefficiencies, and limited insight into teaching quality. These challenges risk slowing down the organization's ability to expand while maintaining its high standard of student learning outcomes.
}

\section{Social relevance}
\instruction{
    The project addresses the broader societal challenge of ensuring equitable access to quality education in an increasingly digital world. As demand for coding literacy grows globally, many children—particularly in rural areas—lack access to qualified instructors and structured learning opportunities. Current manual systems of scheduling, matching, and evaluation are not scalable, limiting the reach and effectiveness of coding education platforms. By automating these processes and introducing data-driven insights, the project not only streamlines operations but also ensures that students are consistently matched with the right instructors and receive high-quality teaching. In doing so, it supports the mission of making coding a universal literacy, reduces barriers to access, and models how education systems can expand sustainably to meet the needs of diverse learners worldwide.
}

\section{Originality/Novelty}
\instruction{
    Solving this problem creates significant value across the broader education and training sector. By automating class scheduling, learner-instructor matching, and performance evaluation, the platform eliminates operational bottlenecks that often slow down digital and hybrid education programs. Educators gain time and clarity to focus on teaching rather than administrative overhead, while learners benefit from being paired with instructors or mentors who best fit their needs—improving engagement, consistency, and outcomes. For institutions and training providers, the centralized dashboard delivers real-time visibility into scheduling, enrollment, payments, and teaching quality, enabling data-driven decision-making and sustainable growth.
    
    Compared with existing solutions, which often rely on fragmented tools like spreadsheets, messaging apps, or generic learning management systems, this platform is purpose-built to address the unique scheduling and performance challenges in skills-based education in an increasingly digital world. While tools such as Zoom, Google Classroom, or Slack are effective for content delivery and communication, they do not solve the deeper challenge of intelligently matching learners to instructors or integrating performance analytics into scheduling. Similarly, traditional CRMs and standalone scheduling tools may handle logistics but lack an educational focus and provide no meaningful insight into teaching outcomes. The value of this project lies in combining these capabilities—matching, scheduling, delivery integration, and analytics—into one cohesive system. Unlike siloed solutions, this platform is designed to scale across institutions and adapt to diverse learning contexts, making it uniquely positioned to drive efficiency and quality in modern education worldwide.
}

\section{CS Contribution}
The project integrates concepts from multiple areas of Computer Science, directly reflecting the knowledge and skills gained through the courses offered at Habib. Key contributions include:

\begin{itemize}
    \item \textbf{Deep Learning:} Applying advanced machine learning techniques to evaluate instructor performance. Models are trained on audio and text data using frameworks such as TensorFlow and PyTorch, leveraging Natural Language Processing (NLP), sentiment analysis, and speech recognition to generate meaningful effectiveness scores.
    
    \item \textbf{Algorithm Design and Analysis:} Developing an intelligent scheduling engine that uses optimization techniques, such as greedy algorithms and genetic algorithms, to minimize conflicts and maximize student--instructor matching efficiency. Algorithmic complexity analysis ensures the system remains efficient at scale.
    
    \item \textbf{Data Structures and Algorithms:} Implementing efficient data management through appropriate structures (heaps, graphs, hash maps) to handle scheduling, student allocation, and query processing. These choices ensure low-latency responses and robust handling of large datasets.
    
    \item \textbf{Web and Mobile Development:} Designing and building a scalable, user-friendly application accessible across devices. This includes developing modular front-end and back-end architectures, integrating APIs (Zoom, scheduling, billing), and ensuring usability through responsive design and intuitive dashboards for instructors, students, and administrators.
    
    \item \textbf{Software Engineering:} Developing a modular, scalable web application that integrates multiple APIs (Zoom, scheduling, billing). Emphasis is placed on maintainability, usability, and meeting non-functional requirements such as reliability and availability.
\end{itemize}

Collectively, these contributions dem   onstrate the application of theoretical knowledge to solve real-world problems, bridging advanced algorithms, deep learning, and scalable software engineering practices.



\section{Scope and Deliverables}
Given the year-long duration and a team of four members, the project scope is quite feasible considering how we will all be working according to our strengths. The system is divided into manageable modules that allow development to continue side by side instead of waiting for each module to finish. The scope includes:

\begin{itemize}
    \item Development of an instructor portal with availability management and performance analytics.
    \item Implementation of a scheduling algorithm that assigns classes automatically while preventing conflicts.
    \item Creation of a student--instructor matching algorithm leveraging computational intelligence.
    \item Integration with the Zoom API for automated link generation and distribution.
    \item Machine learning models for instructor performance analysis based on audio and text data from class recordings.
    \item Administrative dashboard providing reporting, analytics, and payment processing.
\end{itemize}

\textbf{Deliverables:}
\begin{itemize}
    \item Functional web-based platform with separate dashboards for instructors and administrators.
    \item Database schema and implementation for managing schedules, profiles, and performance records.
    \item Smart scheduling engine with optimization-based algorithms (this will include the relevant ML models as well).
    \item Documentation including design artifacts and API specifications.
    \item Deployment-ready system demonstration.
\end{itemize}


\section{Feasibility}
The project can be built with tools and resources we already have or can access.

\subsection*{Datasets}
\begin{itemize}
	\item Zoom video recordings from CodeSchool classes.
	\item Zoom audio recordings for speech and engagement analysis.
	\item Transcribed data for text-based evaluation.
	\item Student and instructor profiles for scheduling and matching.
\end{itemize}

\subsection*{Compute resources}
\begin{itemize}
	\item Each team member has a laptop with enough power for coding and basic tests.
	\item For training heavy models, we will use cloud platforms like Google Colab or AWS.
\end{itemize}

\subsection*{Software}
\begin{itemize}
	\item Machine learning: TensorFlow, PyTorch, Scikit-learn.
	\item Backend: Node.js, Ruby on Rails.
	\item Frontend: React.js, Next.js.
	\item Database: PostgreSQL.
\end{itemize}

\subsection*{Hardware}
\begin{itemize}
	\item Standard laptops are enough for development.
	\item For training models, we can use the university GPU.
	\item If models or data are large, we will use Google Colab Pro, which we may need to buy.
	\item The choice depends on dataset size and how complex our models are.
\end{itemize}

\subsection*{Access}
\begin{itemize}
	\item All required tools are free or open source.
	\item Student credits from cloud services cover large training jobs.
	\item Zoom API provides integration for recordings and class links.
\end{itemize}

\subsection*{Risks and fixes}
\begin{itemize}
	\item If Zoom data is incomplete, we will use stored transcripts and profiles for testing.
	\item If local machines are slow, we will use cloud GPUs.
\end{itemize}

\subsection*{Conclusion}
The project is realistic. The tools are ready, the data is available, and the plan is clear.

\section{Team dynamics}
The project team consists of four members, each bringing complementary expertise across backend development, frontend design, databases, and machine learning. This distribution of roles ensures that the interdisciplinary requirements of the system are well-covered and integration can be achieved smoothly.

\begin{itemize}
    \item \textbf{Naaseh Sajid (ms08085):} Specializes in backend development, API integration, and database management. Skilled in Node.js, Next.js, Ruby on Rails, React, Python, and PostgreSQL. He is responsible for developing the backend services that connect the web application with the database and evaluation modules, as well as maintaining and populating the database.
    
    \item \textbf{Ifrah Chishti (ic08351):} Contributes to both frontend and backend development, with additional expertise in databases and data science. Experienced in Node.js, Python, PostgreSQL, and data science. She is primarily responsible for implementing design and code logic aligned with CodeSchool’s requirements, while also providing backend support alongside Naaseh and Aina.
    
    \item \textbf{Aina Shakeel (as08430):} Focuses on machine learning and computational intelligence. Proficient in Python, TensorFlow, Scikit-learn, and deep learning techniques. She is tasked with developing and integrating the student–instructor matching model, and also supports backend architecture to enable robust data collection for future performance analytics.
    
    \item \textbf{Zain Hatim (zh08343):} Specializes in frontend design and user experience. Skilled in React, React Native, Node.js, and Python. He is responsible for building the frontend interface in React, ensuring usability, simplicity, and an aesthetically appealing design, as well as addressing any UI/UX issues that arise.
\end{itemize}


\section{Tech stack}
The project employs a modern and scalable technology stack, chosen for its robustness and ability to support modular development.

\begin{itemize}
    \item \textbf{Frontend:} React.js / Next.js for building responsive, component-based UIs with smooth state management and routing.
    \item \textbf{Backend:} Ruby on Rails or Next.js (API routes) to implement RESTful services for scheduling, user management, and system logic.
    \item \textbf{Machine Learning:} Python with libraries such as TensorFlow, PyTorch, and SpeechRecognition for audio feature extraction, NLP-based analysis, and building instructor performance models.
    \item \textbf{Database:} PostgreSQL as the primary relational database for storing instructor availability, student profiles, class schedules, and performance metrics.
    \item \textbf{Video Conferencing Integration:} Zoom API for automatic meeting creation and link distribution; Google Meet as a fallback option.
    \item \textbf{Deployment and Version Control:} Git/GitHub for version control, Docker for containerization, and cloud-based hosting (e.g., AWS or Heroku) for deployment and scalability.
\end{itemize}


% References section
\begin{thebibliography}{00}

\bibitem{demszky2023}
D.~Demszky and J.~Liu, ``M-Powering Teachers: Natural Language Processing Powered Feedback Improves 1:1 Instruction and Student Outcomes,'' \emph{EdWorkingPaper}, no.~23-759, May 2023. [Online]. Available: \url{https://edworkingpapers.com/sites/default/files/ai23-759.pdf}

\bibitem{suresh2022}
A.~Suresh, J.~Jacobs, C.~Harty, M.~Perkoff, J.~H. Martin, and T.~Sumner, ``The TalkMoves Dataset: K-12 Mathematics Lesson Transcripts Annotated for Teacher and Student Discursive Moves,'' \emph{arXiv preprint arXiv:2204.09652}, Apr. 2022. [Online]. Available: \url{https://arxiv.org/pdf/2204.09652}

\bibitem{suresh2021}
A.~Suresh, J.~Jacobs, V.~Lai, C.~Tan, W.~Ward, J.~H. Martin, and T.~Sumner, ``Using Transformers to Provide Teachers with Personalized Feedback on their Classroom Discourse: The TalkMoves Application,'' \emph{arXiv preprint arXiv:2105.07949}, Apr. 2021. [Online]. Available: \url{https://arxiv.org/pdf/2105.07949}

\bibitem{sbeity2014}
I.~Sbeity, M.~Dbouk, and H.~Kobeissi, ``Combining the Analytical Hierarchy Process and the Genetic Algorithm to Solve the Timetable Problem,'' \emph{arXiv preprint arXiv:1409.2650}, Sept. 2014. [Online]. Available: \url{https://arxiv.org/pdf/1409.2650}

\bibitem{tan2025}
X.~Tan, G.~Cheng, and M.~H.~Ling, ``Artificial Intelligence in teaching and teacher professional development: A systematic review,'' \emph{Computers and Education: Artificial Intelligence}, vol.~8, p.~100355, 2025. [Online]. Available: \url{https://doi.org/10.1016/j.caeai.2024.100355}

% \bibitem{hosny2013}
% M.~I. Hosny and S.~Fatima, ``A Survey of Genetic Algorithms for the University Timetabling Problem,'' 2013. [Online]. Available: \url{https://www.researchgate.net/publication/258262948_A_Survey_of_Genetic_Algorithms_for_the_University_Timetabling_Problem}

% \bibitem{hooshmand2013}
% S.~Hooshmand, M.~Behshameh, and O.~Hamidi, ``A Tabu Search Algorithm with Efficient Diversification Strategy for High School Timetabling Problem,'' \emph{arXiv preprint arXiv:1309.3285}, Sept. 2013. [Online]. Available: \url{https://arxiv.org/pdf/1309.3285}

\end{thebibliography}

% External advisor undertaking.
\input{external}

\end{document}

%%% Local Variables:
%%% mode: latex
%%% TeX-master: t
%%% End:
