This chapter provides detailed specifications of the system under development.

\section{Project Plan}
\subsection{Objectives}
The goal of this project is to design and implement an Intelligent Scheduling and Performance Analytics Platform for Code School. The big picture is a centralized web-based CMS that automates administrative workflows, enhances scheduling efficiency, and provides AI-driven instructor evaluation reports based on classroom and student progress information. 

Outside of the traditional CMS components, key objectives include: 
\begin{itemize}
    \item Automation: Replace manual scheduling with a centralized, conflict-free scheduling system that takes instructor preferences, timetables, and time zones into account. 
    \item AI Integration: Utilize transcripts of past Zoom recordings to train a NLP-based model to automatically evaluate instructor performance based on classroom environment, taking key performance indicators such as etc into account. 
    \item Integration with external APIs: Enable automated Zoom link generation and integration, along with automated email services to send emails to parents each month regarding class details and student performance.  
    \item Analytics Dashboard: Provide visual analytics for instructors and admins to monitor student progress, teaching quality, and finance-related matters. 
\end{itemize}

\subsection{Tasks and Timelines}
\begin{table}[ht!]
\centering
\renewcommand{\arraystretch}{1.2}
\small
\begin{tabular}{|p{3.3cm}|p{2.8cm}|p{7.2cm}|p{2.5cm}|}
\hline
\textbf{Phase} & \textbf{Duration} & \textbf{Tasks / Deliverables} & \textbf{Responsible Members} \\ \hline

\textbf{1. Requirement Analysis \& Design} &
Oct 2025 -- Nov 2025 &
Collect requirements from \textit{CodeSchool}, define use cases, design ERD \& wireframes, and research relevant APIs/NLP models. &
All members \\ \hline

\textbf{2. Frontend Development} &
Dec 2025 -- mid Jan 2026 &
Build React (Next.js) dashboard for Admin/Instructor, design UI for performance and student modules, and gather feedback on components. &
Zain, Naaseh \\ \hline

\textbf{3. Backend \& Database Setup} &
Jan 2026 -- early Feb 2026 &
Set up PostgreSQL database, develop API endpoints, implement secure authentication \& data management, and review backend design. &
Naaseh, Ifrah, Zain \\ \hline

\textbf{4. ML Integration \& Scheduling Algorithm} &
Feb 2026 &
Develop scheduling algorithm, train NLP model for analytics, and integrate ML with backend. &
Aina, Ifrah \\ \hline

\textbf{5. System Integration \& Testing} &
Mar 2026 &
Integrate frontend, backend, and ML modules; conduct unit/integration tests; verify Zoom API functionality. &
All members \\ \hline

\textbf{6. Documentation \& Final Report} &
Apr 2026 &
Prepare final documentation, reports, and project presentation. &
All members \\ \hline

\end{tabular}
\caption{Project Tasks and Timeline}
\label{tab:tasks_timeline}
\end{table}

\section{Wireframes}
This section presents the initial wireframes designed for the Instructor and Admin dashboards, as well as supporting views for class and instructor management.

\begin{figure}[htbp]
    \centering
    \includegraphics[width=0.9\textwidth]{../images/wireframes/Intructor-Dashboard-Screen.jpg}
    \caption{Instructor Dashboard wireframe.}
    \label{fig:instructor_wireframe}
\end{figure}

\begin{figure}[htbp]
    \centering
    \includegraphics[width=0.9\textwidth]{../images/wireframes/Admin-Dashboard-Screen.jpg}
    \caption{Admin Dashboard wireframe.}
    \label{fig:admin_wireframe}
\end{figure}

\begin{figure}[htbp]
    \centering
    \includegraphics[width=0.9\textwidth]{../images/wireframes/Admin-Classes-Screen.jpg}
    \caption{Admin Classes wireframe.}
    \label{fig:admin_classes_wireframe}
\end{figure}

\begin{figure}[htbp]
    \centering
    \includegraphics[width=0.9\textwidth]{../images/wireframes/Admin-Classes-Specific-Info-1-Screen.jpg}
    \caption{Admin Class Details wireframe.}
    \label{fig:admin_classes_specific_info_wireframe}
\end{figure}

\begin{figure}[htbp]
    \centering
    \includegraphics[width=0.9\textwidth]{../images/wireframes/Admin-Instructors-Screen.jpg}
    \caption{Admin Instructors List wireframe.}
    \label{fig:admin_instructors_wireframe}
\end{figure}

\section{Functional Requirements}

This section details all functions and features of the system. Each requirement is organised into modules with identifiers, priorities, descriptions, detailed requirements, and acceptance criteria.

\subsection{Instructor Management}

\subsubsection{FR-IM-001: Instructor Profile Management}
\textbf{  Priority:} High

\textbf{Description:} The system shall maintain comprehensive instructor profiles including availability, expertise, preferences, and performance history.

\textbf{Detailed Requirements:}
\begin{itemize}
    \item Profile includes: name, email, contact information, country, coding languages taught, preferred age groups, teaching experience level (XP).
    \item Instructors set weekly availability (day, start time, end time) with support for multiple time slots and unavailable periods.
    \item Instructors define class preferences including minimum/maximum classes per month/day.
    \item Profile automatically tracks teaching history, total classes taught, average ratings, and performance metrics.
    \item Instructors may update availability and preferences, with changes effective in the next scheduling cycle.
\end{itemize}

\textbf{Acceptance Criteria:}
\begin{itemize}
    \item Profile form validates required fields and data types.
    \item Availability supports multiple time slots per day.
    \item Historical data is immutable and audit-logged.
\end{itemize}

\subsection{Student and Parent Management}

\subsubsection{FR-SPM-001: Student Registration and Profile Management}
\textbf{Priority:} High  

\textbf{Description:} The system shall maintain detailed student profiles linked to parent/guardian accounts.

\textbf{Detailed Requirements:}
\begin{itemize}
    \item Student profile includes: name, date of birth, current course/module.
    \item Parent profile includes: name, email, contact number, country, billing information and subscription plan.
    \item System automatically calculates student age based on date of birth.
    \item A parent may register and manage multiple students.
    \item System tracks student progress across modules and courses.
\end{itemize}

\textbf{Acceptance Criteria:}
\begin{itemize}
    \item Multiple students can be linked to one parent account.
    \item Automatic and accurate age calculation.
    \item Instructors assigned to a student can access relevant profile data.
\end{itemize}

\subsubsection{FR-SPM-002: Enrollment Management}
\textbf{Priority:} High  

\textbf{Description:} The system shall manage student enrollments including subscription status, attendance, and lesson-level feedback.

\textbf{Detailed Requirements:}
\begin{itemize}
    \item Enrollment links a student to a scheduled class.
    \item Subscription statuses: active, pending, suspended, cancelled.
    \item Attendance recorded for each of four monthly lessons (present/absent).
    \item Lesson-level data stored: project links, code ratings (0--100), Kahoot scores, instructor comments.
    \item System auto-calculates average code rating across lessons.
    \item Next course/module is automatically assigned, however, instructors may override.
\end{itemize}

\textbf{Acceptance Criteria:}
\begin{itemize}
    \item Attendance and performance data editable only by instructors.
    \item Accurate automatic calculation of average code rating.
    \item Notification triggered on subscription status changes.
\end{itemize}

\subsection{Class Scheduling and Assignment}

\subsubsection{FR-CSA-001: Automated Class Scheduling}
\textbf{Priority:} Critical  

\textbf{Description:} The system shall automatically generate monthly class schedules using instructor availability, student demand, and optimization constraints.

\textbf{Detailed Requirements:}
\begin{itemize}
    \item Scheduling algorithm runs monthly after registration close date.
    \item Algorithm considers:
    \begin{itemize}
        \item Instructor availability and min/max class preferences.
        \item Instructor expertise (languages, age groups), XP.
        \item Student preferences (time slots, age, language, class type).
        \item Existing enrollments and progression.
        \item Zoom account availability (2 accounts; Google Meet fallback).
    \end{itemize}
    \item Prevents instructor schedule conflicts.
    \item Optimizes workload balance and student-instructor fit.
    \item Schedule includes class day, time, start date, and 4-week frequency.
\end{itemize}

\textbf{Acceptance Criteria:}
\begin{itemize}
    \item No overlapping instructor class times on the same zoom account.
    \item All active students assigned to classes.
    \item Instructor class count within min/max preferences unless admin override.
    % \item Algorithm completes within 2 hours for up to 200 instructors and 2000 students.
\end{itemize}

\subsubsection{FR-CSA-002: Intelligent Student--Instructor Matching}
\textbf{Priority:} Critical  

\textbf{Description:} The system shall match students with instructors using a machine learning--assisted algorithm.

\textbf{Detailed Requirements:}
\begin{itemize}
    \item Factors considered:
    \begin{itemize}
      \item Student age and instructor preferred age groups.
      \item Instructor expertise level.
      \item Historical student performance (ratings, Kahoot scores).
      \item Instructor teaching effectiveness scores (ML model).
    \end{itemize}
    \item Uses round-robin allocation to ensure fairness.
    \item Prioritizes continuity with previous instructors.
    \item Alerts admin for groups classes that can be potentially merged.
\end{itemize}

\textbf{Acceptance Criteria:}
\begin{itemize}
    \item Matching improves student outcomes relative to random assignment.
    \item No instructor assigned beyond maximum class preference without override.
    \item Special cases flagged for manual review.
    % \item Matching executes in under 30 minutes.
\end{itemize}

\subsubsection{FR-CSA-003: Schedule Publication and Notification}
\textbf{Priority:} High  

\textbf{Description:} The system shall publish finalized schedules and notify all stakeholders.

\textbf{Detailed Requirements:}
\begin{itemize}
    \item Instructor dashboard displays monthly class list with:
    \begin{itemize}
        \item Student details (name, age, country)
        \item Zoom links
    \end{itemize}
    \item Start-of-month emails sent to instructors with full schedule and meeting links.
    % \item Students/parents receive schedule email, calendar invite, instructor bio, and meeting links.
\end{itemize}

\textbf{Acceptance Criteria:}
\begin{itemize}
    \item All notifications sent within 24 hours of schedule finalization.
    \item Emails contain complete and correct details.
    \item Delivery failures logged for admin review.
\end{itemize}

\subsection{Video Conferencing Integration}

\subsubsection{FR-VCI-001: Zoom Meeting Generation}
\textbf{Priority:} Critical  

\textbf{Description:} The system shall automatically generate Zoom meetings for all scheduled classes.

\textbf{Detailed Requirements:}
\begin{itemize}
    \item Integrates with Zoom API to create recurring meetings.
    \item Each meeting includes unique ID, password, join link.
    \item Meeting settings:
    \begin{itemize}
        \item Waiting room enabled
        \item Recording enabled (cloud)
        \item Chat disabled
        \item Duration: 60 minutes
    \end{itemize}
    \item Load balancing across 2 Zoom accounts; fallback to Google Meet.
    \item 15 minute buffer maintained before and after each class to allow for overtime.
\end{itemize}

\textbf{Acceptance Criteria:}
\begin{itemize}
    \item Meetings created for 100\% of online scheduled classes.
    \item No double-booking of Zoom accounts.
    \item Links accessible to enrolled students and instructors.
\end{itemize}

% \subsubsection{FR-VCI-002: Recording Management}
% \textbf{Priority:} Medium  

% \textbf{Description:} The system shall manage storage and access to class recordings.

% \textbf{Detailed Requirements:}
% \begin{itemize}
%     \item Automatic download of recordings after each class.
%     \item Secure encrypted file storage.
%     \item Access control: instructors access own recordings; admins full access.
%     \item Audio extraction for ML analysis; transcription generated.
% \end{itemize}

% \textbf{Acceptance Criteria:}
% \begin{itemize}
%     \item Recordings downloaded successfully after class.
%     \item Transcription and audio extraction complete within 24 hours.
%     \item Access restricted to authorised users.
% \end{itemize}

\subsection{Course and Curriculum Management}

\subsubsection{FR-CCM-001: Course Catalog Management}
\textbf{Priority:} Medium  

\textbf{Description:} The system shall maintain a structured catalog of courses and modules.

\textbf{Detailed Requirements:}
\begin{itemize}
    \item Catalog includes course name, description, age range, prerequisites.
    \item Each course contains sequenced modules with learning objectives.
    \item Courses tagged by coding language (Scratch, Python, etc.).
    \item Flashcards availability indicator.
    \item Curriculum version control maintained.
\end{itemize}

\textbf{Acceptance Criteria:}
\begin{itemize}
    \item Admins can create/update/archive courses.
    \item Linear module progression enforced unless overridden.
    \item Catalog searchable and filterable.
\end{itemize}

\subsubsection{FR-CCM-002: Class Content Management}
\textbf{Priority:} Medium  

\textbf{Description:} The system shall provide instructors with all necessary curriculum materials.

\textbf{Detailed Requirements:}
\begin{itemize}
    \item Lesson plans, project demos, and supplemental resources.
    \item Kahoot links for end-of-month assesments.
    \item Flashcards accessible where available.
    \item Version control ensuring updated materials.
\end{itemize}

\textbf{Acceptance Criteria:}
\begin{itemize}
    \item All materials accessible from instructor dashboard.
    \item Materials organized by course and module.
\end{itemize}

\subsection{Performance Evaluation and Analytics}

\subsubsection{FR-PEA-001: Instructor Performance Evaluation via ML}
\textbf{Priority:} High  

\textbf{Description:} The system shall automatically evaluate instructor performance using ML-based analysis of class recordings.

\textbf{Detailed Requirements:}
\begin{itemize}
    \item ML model analyzes audio and transcripts.
    \item Evaluation metrics scored 1--5:
    \begin{itemize}
        \item Introduction clarity
        \item Recap effectiveness
        \item Exercise quality
        \item Energy and vocal engagement
        \item Patience
        \item Class culture
        \item Student attention
        \item Concept review
        \item Curriculum fidelity
        \item Lesson pacing
    \end{itemize}
    \item Weighted overall teaching effectiveness score generated.
    \item Evaluations done at the end of every working month.
\end{itemize}

\textbf{Acceptance Criteria:}
\begin{itemize}
    % \item ML model achieves minimum 75\% accuracy.
    % \item Evaluations available within 48 hours.
    \item Recommendations are actionable.
\end{itemize}

\subsubsection{FR-PEA-002: Manual Class Evaluation}
\textbf{Priority:} Medium  

\textbf{Description:} Admins may evaluate classes manually for quality assurance.

\textbf{Detailed Requirements:}
\begin{itemize}
    \item Admins select classes for manual review.
    \item Evaluation rubric matches ML system.
    \item Manual evaluations can override ML scores.
\end{itemize}

\textbf{Acceptance Criteria:}
\begin{itemize}
    \item Evaluations timestamped and attributed.
    % \item Completed within 5 business days.
\end{itemize}

\subsubsection{FR-PEA-003: Performance Reporting and Analytics}
\textbf{Priority:} Medium  

\textbf{Description:} The system shall provide analytics dashboards for instructors and administrators.

\textbf{Detailed Requirements:}
\begin{itemize}
    \item Instructor dashboard metrics:
    \begin{itemize}
        \item Performance trends (3, 6, 12 months)
        % \item Peer comparison (anonymised)
        \item Evaluation metric breakdown
        \item Student progression statistics
        \item Overall XP and language specific XP
    \end{itemize}
    \item Admin dashboard metrics:
    \begin{itemize}
        \item Performance distributions
        \item Top/bottom performers
        \item Trends by country/language/age group
        \item Class capacity utilization
    \end{itemize}
\end{itemize}

\textbf{Acceptance Criteria:}
\begin{itemize}
    % \item Dashboards load within 3 seconds.
    \item Data visualizations clear and interactive.
    \item Reports exportable to PDF/CSV.
\end{itemize}

\subsection{Communication and Notifications}

\subsubsection{FR-CN-001: Automated Email Communications}
\textbf{Priority:} High  

\textbf{Description:} The system shall send automated emails at key lifecycle stages.

\textbf{Detailed Requirements:}
\begin{itemize}
    \item Start-of-month instructor email: schedule, Zoom links, roster.
    \item End-of-month instructor email: hours verification, payments, performance summary.
    \item Start-of-month parent email: schedule, calendar invite, instructor bio, Zoom links.
    % \item Parent renewal reminders 7 days before month end; second reminder 2 days before.
    % \item Class reminders to students 1 hour before class.
\end{itemize}

\textbf{Acceptance Criteria:}
\begin{itemize}
    \item Emails sent on schedule.
    \item Content uses personalized placeholders.
\end{itemize}

\subsubsection{FR-CN-002: In-App Notifications}
\textbf{Priority:} Medium  

\textbf{Description:} The system shall provide real-time in-app notifications.

\textbf{Detailed Requirements:}
\begin{itemize}
    \item Notification types: schedule updates, evaluations, payments, announcements.
    \item Unread count indicator.
    \item Notification history maintained for 30 days.
\end{itemize}

\textbf{Acceptance Criteria:}
\begin{itemize}
    % \item Notifications appear within 30 seconds.
    \item Unread count updates in real-time.
\end{itemize}

\subsection{Payment Management}

\subsubsection{FR-PM-001: Automated Payment Calculation}
\textbf{Priority:} High  

\textbf{Description:} The system shall calculate instructor payments based on completed classes.

\textbf{Detailed Requirements:}
\begin{itemize}
    \item Payment depends on:
    \begin{itemize}
        \item Number of classes taught
        \item Attendance verification
        \item Base rate per class
        \item Bonuses (academic and admin)
    \end{itemize}
    \item Payment Request Form generated containing: instructor name, month, classes taught, hours, base amount, bonuses.
\end{itemize}

\textbf{Acceptance Criteria:}
\begin{itemize}
    \item Payments match manual verification.
    \item PRF generated by month end.
\end{itemize}

\subsubsection{FR-PM-002: Payment Processing Workflow}
\textbf{Priority:} Medium  

\textbf{Description:} The system shall manage payment review, approval, and tracking.

\textbf{Detailed Requirements:}
\begin{itemize}
    \item Finance team reviews PRF before approval.
    \item Payment statuses: pending, approved, sent.
    \item Instructors notified upon payment.
    \item Payment history stored for 12 months.
\end{itemize}

\textbf{Acceptance Criteria:}
\begin{itemize}
    \item Status changes logged with timestamps.
    \item Reports exportable for accounting.
\end{itemize}

\subsection{Administrative Functions}

\subsubsection{FR-AF-001: User Management}
\textbf{Priority:} High  

\textbf{Description:} The system shall provide role-based user management for all system users.

\textbf{Detailed Requirements:}
\begin{itemize}
    \item Admins can create/update users (instructors, admins).
    \item Roles:
    \begin{itemize}
        \item Super Admin: full access
        \item Operations Admin: scheduling, enrollment
        \item Finance Admin: payments
        \item Evaluation Admin: performance reviews
        \item Instructor: class-level access
    \end{itemize}
    % \item Password reset with secure tokens.
\end{itemize}

\textbf{Acceptance Criteria:}
\begin{itemize}
    \item Role permissions enforced across system.
    % \item Reset emails sent within 5 minutes.
\end{itemize}

\subsubsection{FR-AF-002: Data Export and Reporting}
\textbf{Priority:} Medium  

\textbf{Description:} The system shall support export of key datasets.

\textbf{Detailed Requirements:}
\begin{itemize}
    \item Export formats: CSV, XLSX, PDF.
    \item Exportable data:
    \begin{itemize}
        % \item Student roster
        % \item Instructor directory
        % \item Schedules and enrollments
        % \item Attendance records
        \item Performance analytics
        \item Payment summaries
    \end{itemize}
\end{itemize}

\textbf{Acceptance Criteria:}
\begin{itemize}
    % \item Exports complete within 2 minutes for 10,000 records.
    \item Sheets sync updates within 10 minutes.
\end{itemize}

\subsubsection{FR-AF-003: System Configuration}
\textbf{Priority:} Low  

\textbf{Description:} The system shall allow configuration of global system settings.

\textbf{Detailed Requirements:}
\begin{itemize}
    \item Configurable parameters:
    \begin{itemize}
        \item Registration deadlines
        \item Email templates
        \item Payment rates and bonus amount
        \item Zoom account credentials
        \item Evaluation metric weights
    \end{itemize}
    \item All changes logged with timestamp and user.
    \item Critical settings require dual approval.
\end{itemize}

\textbf{Acceptance Criteria:}
\begin{itemize}
    \item Invalid configurations rejected with error messages.
    \item All changes correctly recorded in logs.
\end{itemize}

% This section presents detailed use cases for the system. Each use case includes actors, descriptions, data inputs/outputs, system responses, and additional notes.


\section{Non-functional Requirements}

\subsection{Performance Requirements}
\begin{itemize}
    \item \textbf{Response Time:} The system should respond to instructor or admin requests (e.g., loading class data, managing students) within a reasonable time (3–5 seconds) under normal network conditions.
    \item \textbf{Concurrent Users:} The system should support all current instructors concurrently without noticeable performance degradation.
\end{itemize}

\subsection{Reliability \& Availability}
\begin{itemize}
    \item \textbf{Uptime:} The system should maintain availability during the working hours of each instructor.
    \item \textbf{Error Handling:} In case of unexpected errors (e.g., database disconnection), the system should display user-friendly error messages and attempt automatic recovery.
\end{itemize}

\subsection{Security Requirements}
\begin{itemize}
    \item \textbf{Authentication:} Only authorised users (admins and instructors) should be able to log in, verified through stored usernames and passwords.
    \item \textbf{Access Control:} Instructors should only have access to their assigned classes, while admins should have full access to all data.
    \item \textbf{Data Protection:} Sensitive information (passwords, student data, grades) should be encrypted in storage and during transmission (HTTPS).
\end{itemize}

\subsection{Usability Requirements}
\begin{itemize}
    \item \textbf{Interface Simplicity:} The interface should be intuitive and easy to navigate, requiring minimal training for instructors and admins.
    \item \textbf{Accessibility:} The system should use readable fonts, consistent layouts, and colour contrast suitable for all users.
    \item \textbf{Feedback:} The system should provide clear feedback for user actions (e.g., confirmation messages on successful operations).
\end{itemize}

\subsection{Maintainability}
\begin{itemize}
    \item \textbf{Modular Design:} The codebase should follow a modular architecture (e.g., MVC or component-based structure) to make updates easier.
    \item \textbf{Documentation:} The project should include relevant developer documentation for database schema, APIs, and code structure.
\end{itemize}

\subsection{Portability}
\begin{itemize}
    \item \textbf{Browser Compatibility:} The CMS should function correctly on modern browsers (Chrome, Edge, Firefox).
    \item \textbf{Device Compatibility:} The system should be optimised for desktop use.
\end{itemize}

\subsection{Compliance \& Ethical Considerations}
\begin{itemize}
    \item \textbf{Data Privacy:} The system must comply with general data protection principles (no unauthorised sharing of student information).
    \item \textbf{Content Integrity:} The system must prevent unauthorised alteration or deletion of class records or grades (ideally via efficient design where no user can supersede their roles).
\end{itemize}

\section{External Interfaces}

We expect every project to have at least of the following subsections. This section must be aligned with your project deliverables. Please consult with your project supervisor regarding which of the following section(s) you should include in your report

\subsection{User Interfaces}
This section includes our mockup screens and briefly explains them.

\subsection{Application Program Interface (API)}
This section describes the library or API interface to our system.

\subsection{Hardware/Communication Interfaces}
This section describes our project's specific hardware/network interfaces.

\section{Use Cases}
This section presents detailed use cases of our system.

\subsection{Use Case Diagram}
\begin{center}
    \includegraphics[width=1\textwidth]{../images/UseCaseDiagram.jpg}
\end{center}

\subsection{Use Case 1: Generate Monthly Class Schedule}

\begin{tabular}{|p{3.5cm}|p{11cm}|}
\hline
\textbf{Field} & \textbf{Details} \\ \hline
Actors & Administrator (primary), Scheduling Algorithm (system), ML Matching Algorithm (system), Instructors, Students, Zoom API \\ \hline
Description & The administrator initiates the monthly scheduling process. The system retrieves instructor availability and student enrollment data, applies optimization algorithms, and generates a conflict-free schedule. The ML matching algorithm assigns students to suitable instructors based on age, XP, past performance, and learning preferences. Zoom meeting links are automatically generated and distributed via email and dashboards. \\ \hline
Data & \textbf{Input:} Instructor availability, student enrollment records, instructor XP and performance data, Zoom account availability. \\
& \textbf{Output:} Published monthly schedule, instructor-student assignments, Zoom links, notification emails, dashboard updates. \\ \hline
Stimulus & Administrator clicks the ``Generate Schedule'' button at the end of the month. \\ \hline
Response & System generates the schedule, validates conflicts, creates Zoom links, publishes data to dashboards, and sends notifications. Admin receives a summary with flagged edge cases. \\ \hline
Comments & Handles small class sizes ($<3$ students), Zoom capacity constraints (2 accounts, Google Meet fallback), XP-based instructor prioritization, and round-robin workload balancing. \\ \hline
\end{tabular}

\subsection{Use Case 2: Set Weekly Availability}

\begin{tabular}{|p{3.5cm}|p{11cm}|}
\hline
\textbf{Field} & \textbf{Details} \\ \hline
Actors & Instructor (primary), Scheduling Algorithm (system) \\ \hline
Description & Instructor logs in and sets weekly availability by selecting days, time ranges, and teaching preferences (min/max classes, languages, age groups). System validates entries and stores them for the next scheduling cycle. \\ \hline
Data & \textbf{Input:} Weekly availability (days/times), min/max classes, preferred languages, unavailable days. \\
& \textbf{Output:} Stored availability record, on-screen confirmation, email confirmation. \\ \hline
Stimulus & Instructor clicks ``Edit Availability'' or responds to reminder before the 10th of the month. \\ \hline
Response & System displays existing availability, validates changes, saves data, and confirms update. Changes apply next month. \\ \hline
Comments & Availability must be submitted by the 10th. Supports vacation ranges and multi-slot days. Each slot must be at least 1 hour. Mobile-friendly interface. \\ \hline
\end{tabular}

\subsection{Use Case 3: Submit Class Evaluation}

\begin{tabular}{|p{3.5cm}|p{11cm}|}
\hline
\textbf{Field} & \textbf{Details} \\ \hline
Actors & Instructor (primary), Students, Parents (receive notifications), Payment System \\ \hline
Description & After each class, instructor fills an evaluation form with attendance, project links, code ratings, comments, and for lesson 4: Kahoot score and recommended next module. \\ \hline
Data & \textbf{Input:} Attendance, project link, code rating (1--100), comments, Kahoot score (lesson 4), recommended course/module. \\
& \textbf{Output:} Saved evaluation record, updated progress metrics, parent email (lesson 4), verified teaching hours for payments. \\ \hline
Stimulus & Instructor clicks ``Pending Evaluations'' or selects a completed class. \\ \hline
Response & System loads roster, auto-saves draft, validates entries, stores data, updates student dashboards, and sends parent email at month end. \\ \hline
Comments & Must be submitted within 24 hours. Late submissions flagged. Students marked absent cannot receive ratings. Evaluations locked after 24 hours unless admin unlocks. \\ \hline
\end{tabular}

\subsection{Use Case 4: Match Students to Instructors}

\begin{tabular}{|p{3.5cm}|p{11cm}|}
\hline
\textbf{Field} & \textbf{Details} \\ \hline
Actors & ML Matching Algorithm (primary), Administrator (review), Instructor/Student Profiles \\ \hline
Description & ML algorithm computes optimal student–instructor pairings based on age fit, expertise alignment, performance data, continuity preference, and workload fairness. \\ \hline
Data & \textbf{Input:} Student profiles, instructor profiles, historical data, performance records. \\
& \textbf{Output:} Compatibility scores, assignment list, workload distribution report, flagged cases. \\ \hline
Stimulus & Automatically triggered during scheduling after time slots are fixed. \\ \hline
Response & System generates all pairings, flags low-confidence matches ($<70\%$), and produces summary reports. \\ \hline
Comments & Prioritizes student continuity, flags small class sizes, handles new instructors with limited history, and retrains model periodically using outcome data. \\ \hline
\end{tabular}

\subsection{Use Case 5: Access Zoom Link}

\begin{tabular}{|p{3.5cm}|p{11cm}|}
\hline
\textbf{Field} & \textbf{Details} \\ \hline
Actors & Instructor (primary), Students/Parents, Zoom API \\ \hline
Description & User logs in and opens their class schedule. System displays Zoom links. ``Join Class'' button becomes active 15 minutes before class. \\ \hline
Data & \textbf{Input:} User authentication, class ID, timestamp. \\
& \textbf{Output:} Zoom link, meeting ID/password, reminder email. \\ \hline
Stimulus & User navigates to ``My Schedule''. \\ \hline
Response & System verifies time, logs link access, redirects to Zoom, and sends reminder if user hasn't joined after 5 minutes. \\ \hline
Comments & Links are recurring for all 4 lessons. System tracks join-time for attendance verification. Google Meet link generated as fallback if Zoom accounts reach capacity. \\ \hline
\end{tabular}

\subsection{Use Case 6: View Performance Report}

\begin{tabular}{|p{3.5cm}|p{11cm}|}
\hline
\textbf{Field} & \textbf{Details} \\ \hline
Actors & Instructor (primary), ML Evaluation System, Administrator \\ \hline
Description & Instructor views monthly performance report generated by ML analysis of class recordings, including score breakdown, trends, peer comparison, and recommendations. \\ \hline
Data & \textbf{Input:} Recording audio, transcript, lesson plan, evaluation rubric. \\
& \textbf{Output:} Metric scores (1--5), weighted score, trend graphs, peer comparison, recommendations, ML confidence score. \\ \hline
Stimulus & Instructor opens ``My Performance'' or clicks email notification. \\ \hline
Response & System loads report with interactive visualizations, highlights improvements/regressions, and allows PDF download. \\ \hline
Comments & ML uses audio + NLP analysis; low-confidence reports reviewed manually. Scores affect instructor bonuses and class assignment priority. \\ \hline
\end{tabular}

\subsection{Use Case 7: Process Instructor Payments}

\begin{tabular}{|p{3.5cm}|p{11cm}|}
\hline
\textbf{Field} & \textbf{Details} \\ \hline
Actors & Payment System (primary), Administrator, Instructor \\ \hline
Description & At month-end, system calculates payments using attendance, evaluation scores, and Zoom logs. Generates Payment Request Form (PRF) for admin approval. \\ \hline
Data & \textbf{Input:} Attendance records, instructor rate, evaluation scores, Zoom hours. \\
& \textbf{Output:} PRF (PDF), payment status updates, instructor email. \\ \hline
Stimulus & Auto-triggered on the 1st of the month or manually by admin. \\ \hline
Response & System calculates base pay + bonuses, generates PRF, notifies admin, and after approval notifies instructor. \\ \hline
Comments & Excludes classes with zero attendance. Bonuses stack. Adjustments require justification. Multi-currency support included. Payment statuses: Pending, Approved, Sent. \\ \hline
\end{tabular}

\section{Datasets}
This section describes the specific dataset(s) used to build our system. An appropriate snapshot of the dataset(s) is also included. Futher details, when needed, are presented in the appendix.

\section{System Diagram}
This diagram gives a high-level view of the different components of our system and the interactions between them. Each component and the particular tools/technologies/libraries used to build it are described.

\subsection{Users and Roles}

Each role interacts with the system via the Instructor Dashboard or Admin Dashboard. 

\begin{itemize}
    \item \textbf{Instructor:} Views class schedules, receives performance evaluations, manages teaching logs, updates availability.
    \item \textbf{Evaluation Admin:} Reviews AI-based instructor evaluations and supplements them with manual notes.
    \item \textbf{Operations Admin:} Adds new students and instructors to the system. Oversees class schedules and instructor-student assignments.
    \item \textbf{Finance Admin:} Manages payment processing and bonuses.
    \item \textbf{Curriculum Team:} Updates course content and tracks which modules are active for upcoming sessions.
\end{itemize}

\subsection{Web Application (Frontend)}
Both dashboards communicate with the backend using REST API calls for real-time data synchronization.
\begin{itemize}
    \item \textbf{Instructor Dashboard:} Displays assigned classes, evaluation reports, and class information (student information, Zoom links, instructor guides, etc.). It pulls class and evaluation data through REST APIs.
    \item \textbf{Admin Dashboard:} Serves as the central control panel for operational oversight. Displays analytics, scheduling summaries, and payment status by fetching data from backend services.
\end{itemize}

\subsection{Backend Server}

Each backend module communicates through internal APIs and interacts with the database to store or retrieve data.

\begin{itemize}
    \item \textbf{User Management:} Handles authentication and role-based access control.
    \item \textbf{Matching Engine:} Implements the ML-based student-instructor matching algorithm using factors such as age, instructor experience, and class type.
    \item \textbf{Scheduling Engine:} Prevents conflicts and ensures equitable workload distribution across instructors.
    \item \textbf{Performance Evaluation Service:} Processes Zoom transcripts via the NLP pipeline to assess teaching quality.
    \item \textbf{Payment Processing:} Calculates and updates instructor payments and bonuses after validations.
    \item \textbf{Notification \& Email Service:} Sends automated notifications (e.g., new class assigned, evaluation completed).
    \item \textbf{Student Management:} Tracks student progress according to CSPK coding journey and updates class modules accordingly.
    \item \textbf{Curriculum Management:} Organizes teaching guides for instructors and classes according to schedules. Allows uploading of new/updated guides.
    \item \textbf{Student Evaluation Module:} Factors in instructor comments regarding student progress when updating the current module.
    \item \textbf{Zoom Meeting Generation:} Generates Zoom links based on the monthly schedule.
    \item \textbf{Instructor Management:} Creates instructor profiles and tracks KPIs such as total hours taught, teaching XP, and instructor proficiencies.
\end{itemize}

\subsection{Database Layer and External Integrations}

These integrations ensure seamless communication between the teaching environment and the backend automation processes.

\subsection*{Databases}
\begin{itemize}
    \item \textbf{Instructors:} Stores instructor profiles, credentials, and availability.
    \item \textbf{Students/Parents:} Stores enrollment data and billing information.
    \item \textbf{Payments:} Logs payment calculations, status, and transactions.
    \item \textbf{Evaluations/Performance Data:} Stores both AI-generated and manual evaluations for longitudinal performance tracking.
    \item \textbf{Schedules \& Zoom Links:} Stores class schedules and meeting information.
    \item \textbf{Users \& Roles:} Stores permissions for each user.
    \item \textbf{Classes/Modules/Resources:} Stores lesson plans, code reviews, and supplementary resources.
\end{itemize}

\subsection*{External Integrations}
\begin{itemize}
    \item \textbf{Zoom API:} Generates and manages meeting links for classes.
    \item \textbf{Email API:} Sends class schedules and reminders.
    \item \textbf{Cloud Storage (Dropbox):} Stores Zoom recordings used for AI evaluation.
    \item \textbf{NLP + Audio Processing Model:} Converts class audio to text and generates performance insights stored in the evaluation database.
\end{itemize}

\subsection*{Data Flow}
The data flow begins when operations finalize the monthly student list. The Scheduling Engine generates class schedules and the system’s Matching Engine pairs students and instructors, storing assignments in the database.  Corresponding Zoom links are generated and stored via API integration. Instructors access this information through the Instructor Dashboard, while administrators can monitor system-wide metrics in real time via the Admin Dashboard. After each session, Zoom recordings are processed through the AI Evaluation Service, producing performance analytics that are reviewed by the Evaluation Team before being logged into the instructor’s profile. This data then feeds into the Finance Module for payment calculation and bonus allocation. 


\begin{figure}[htbp]
    \centering
    \includegraphics[width=1.0\textwidth]{../images/TeachWise Use Case-driven System Block Diagram (1).png}
    \caption{TeachWise System Block Diagram.}
    \label{fig:teachwise_system_block_diagram}
\end{figure}