This chapter provides detailed specifications of the system under development.

\section{Functional Requirements}

This section describes each function/feature provided by our system. These functions are logically grouped into modules based on their purpose/users/mode of operations etc (as per our system). A functional hierarchy may look like:
\begin{outline}
  \1 Module 1:
  \2 Function 1:
  \2 Function 2:
  \3 Sub Function 1
  \3 Sub Function 2
  \1 Module 2:
  \2 Function 1:
  \2 Function 2:
  \1 .........
\end{outline}

% --- The above is to be modified as per your project, e.g. a flat list if your system has limited functional requirements.

\section{Non-functional Requirements}

This sections mentions the specific non-functional requirements of our system. These generally address performance, scalability, safety, availability, deployment etc.

\section{External Interfaces}

We expect every project to have at least of the following subsections. This section must be aligned with your project deliverables. Please consult with your project supervisor regarding which of the following section(s) you should include in your report

\subsection{User Interfaces}
This section includes our mockup screens and briefly explains them.

\subsection{Application Program Interface (API)}
This section describes the library or API interface to our system.

\subsection{Hardware/Communication Interfaces}
This section describes our project's specific hardware/network interfaces.

\section{Use Cases}
This section presents detailed use cases of our system.

\section{Datasets}
This section describes the specific dataset(s) used to build our system. An appropriate snapshot of the dataset(s) is also included. Futher details, when needed, are presented in the appendix.

\section{System Diagram}
This diagram gives a high-level view of the different components of our system and the interactions between them. Each component and the particular tools/technologies/libraries used to build it are described.