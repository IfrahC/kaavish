
This chapter provides important artifacts related to design of our project.

\section{Software Design}

This section offers a thorough description of the system's architecture. User management, class scheduling, curriculum delivery, performance evaluation, payments, and external integrations are all handled by interconnected subsystems that make up the TeachWise platform. The following describes each major class and subsystem, highlighting its functions, data characteristics, and relationships with other system elements.

\subsection{User Management SubSystem}

\begin{figure}[htbp]
    \centering
    \includegraphics[width=1.0\textwidth]{../images/class_diagram.png}
    \caption{Class Diagram of the TeachWise system.}
    \label{fig:class_diagram}
\end{figure}

\subsubsection{User}
For every platform actor, the \texttt{User} class serves as the primary identity object. It stands for the parent entity from which the fundamental characteristics of \texttt{Instructor}, \texttt{Admin}, and \texttt{Student} are derived. Account-level data, including name, role, and email, is stored here. It supports the fundamental operations required for platform authentication and profile maintenance:
\begin{itemize}
    \item \texttt{login() / logout()} - authentication workflow
    \item \texttt{updateProfile()} - modify name, role, or contact information
\end{itemize}
All user types share uniform identity management logic and access control mechanisms thanks to this abstraction.

\subsubsection{Instructor}
The \texttt{Instructor} class represents the teachers registered on the platform. It stores identity attributes (\texttt{instructor\_id, user\_id}) and a calculated performance metric (\texttt{avg\_eval}). Instructors are associated with:
\begin{itemize}
    \item \textbf{Professional metadata:} country, coding languages, teaching experience
    \item \textbf{Instructional KPIs:} teachingXP, totalHoursTaught, averageRating
    \item \textbf{Class associations:} all classes taught historically and upcoming
    \item \textbf{Performance tracking:} linked evaluations from the Performance Evaluation subsystem
\end{itemize}
Key behaviors:
\begin{itemize}
    \item \texttt{setAvailability()} - defines when the instructor can teach
    \item \texttt{updatePreferences()} - updates language preferences
    \item \texttt{viewSchedule()} - shows classes assigned for the week or month
    % \item \texttt{submitEvaluation()} - submit lesson evaluations
    \item \texttt{viewPerformanceReport()} - access performance analytics
\end{itemize}
The \texttt{Instructor} class is central to scheduling, matching, evaluation, and payment workflows.

\paragraph{Instructor Availability}
This class captures repeated weekly availability patterns. Each record specifies month, day of week, and time range.
\begin{itemize}
    \item \texttt{validate()} - checks times for logical consistency
    \item \texttt{getTimeSlots()} - generates discrete teachable times for scheduling
\end{itemize}
Used extensively by the \texttt{SchedulingEngine}.

\subsubsection{Admin}
Admins are system controllers responsible for operational oversight. Attributes include \texttt{admin\_type} (finance, operations, curriculum, etc.). Admins manage:
\begin{itemize}
    \item User accounts
    \item Schedule generation
    \item Payment processing
    \item Evaluation reviews
    \item Report exports
\end{itemize}

\subsubsection{Parent}
The \texttt{Parent} class represents guardians managing one or more students. Stores:
\begin{itemize}
    \item Contact details, country, billing information, subscription type
\end{itemize}
% Capabilities:
% \begin{itemize}
%     \item Register students
%     \item View child's schedule
%     \item Receive notifications about sessions, billing, or progress
% \end{itemize}

\subsubsection{Student}
\texttt{Student} class stores:
\begin{itemize}
    \item Personal details, date of birth, dynamically calculated age
    \item Current course/module
    \item Parent association
    \item Enrollment records
\end{itemize}
Functionalities:
\begin{itemize}
    \item \texttt{calculateAge()} - update age at runtime
    \item \texttt{updateProgress()} - mark module progress
    \item \texttt{getEnrollments()} - retrieve associated enrollments
\end{itemize}

\subsection{Class And Enrollment SubSystem}

\subsubsection{Class}
Represents a scheduled lesson session. Contains:
\begin{itemize}
    \item Time-based details: \texttt{dayOfWeek, classTime, startDate}
    \item Instructor assignment
    \item Course and module linkage
    \item Zoom integrations: \texttt{link, meetingId, password}
\end{itemize}
Key functions:
\begin{itemize}
    \item \texttt{generateZoomLink()} - integrates with ZoomAPI
    \item \texttt{getClassDetails()} - returns metadata
    \item \texttt{getEnrolledStudents()} - retrieves enrolled students
\end{itemize}

\subsubsection{Enrollment}
Represents a student’s participation in a specific class.
\begin{itemize}
    \item Records enrollment date, subscription status, link to lessons
    \item Functions:
    \begin{itemize}
        \item \texttt{updateStatus()} - modify subscription state
        \item \texttt{getLessons()} - retrieve lesson information
        % \item \texttt{calculateAverageRating()} - computes end of month performance
    \end{itemize}
\end{itemize}

\subsubsection{Lesson}
Represents individual teaching instances under an enrollment.
\begin{itemize}
    \item Stores: attendance, project submissions, code quality, Kahoot scores, instructor comments
    \item Functions:
    \begin{itemize}
        \item \texttt{recordAttendance()}
        \item \texttt{submitEvaluation()}
        % \item \texttt{validateRating()}
    \end{itemize}
\end{itemize}

\subsection{Curriculum SubSystem}

\subsubsection{Course}
Defines the full curriculum structure. Contains course objectives, age range, coding language, prerequisites, flashcards, versioning. Functions:
\begin{itemize}
    \item \texttt{getModules()}
    % \item \texttt{updateCurriculum()}
    \item \texttt{isPrerequisiteMet()}
\end{itemize}

\subsubsection{Module}
Divides courses into structured units. Stores sequence number, objectives, lesson plan paths, Kahoot and flashcard links. Functions:
\begin{itemize}
    \item \texttt{getContent()}
    \item \texttt{updateCurriculum()}
    \item \texttt{getNextModule()}
\end{itemize}

\subsection{Scheduling SubSystem}

\subsubsection{Schedule}
Represents generated monthly or weekly schedules.
\begin{itemize}
    \item Tracks month, year, generation date, status, associated classes
    \item Functions:
    \begin{itemize}
        \item \texttt{generateMonthlySchedule()}
        \item \texttt{publishSchedule()}
        \item \texttt{getClasses()}
    \end{itemize}
\end{itemize}

\subsubsection{Scheduling Engine}
Automates class scheduling using instructor availability, student demand, and administrative constraints.
\begin{itemize}
    \item \texttt{optimizeSchedule()} - conflict-free schedules
    \item \texttt{checkConflicts()} - overlapping class detection
    \item \texttt{balanceWorkload()} - fair instructor assignments
    \item \texttt{allocateZoomAccounts()} - prevent account collisions
\end{itemize}

\subsubsection{Matching Engine}
Automates student-instructor pairing:
\begin{itemize}
    \item TeachingXP
    % \item Teaching style
    \item Continuity preferences
    \item Functions:
    \begin{itemize}
        \item \texttt{calculateCompatibility()}
        \item \texttt{matchStudentToInstructor()}
        % \item \texttt{prioritizeContinuity()}
        \item \texttt{generateRecommendations()}
    \end{itemize}
\end{itemize}

\subsection{Performance Evaluation SubSystem}

\subsubsection{Performance Evaluation}
Stores granular metrics of instructor performance: introduction, recap, exercises, energy, patience, classroom culture, attention checks, concept review, curriculum fidelity, pacing. Computes \texttt{overallScore} and generates performance reports.

\subsubsection{MLEvaluationModel}
ML-based subsystem analyzing class recordings:
\begin{itemize}
    \item \texttt{analyzeRecording()}
    \item \texttt{extractTranscript()}
    \item \texttt{scoreMetrics()}
    \item \texttt{generateRecommendations()}
\end{itemize}

\subsection{Payment SubSystem}
Manages payroll for instructors: teaching hours, bonuses, total payout, payment status. Functions:
\begin{itemize}
    \item \texttt{calculatePayment()}
    \item \texttt{generatePRF()}
    \item \texttt{approvePayment()}
\end{itemize}

\subsection{External Integration}

\subsubsection{Zoom API}
Handles integration with Zoom services: meeting creation, recording retrieval, account availability checks.

% \subsubsection{Notification Service}
% Centralizes notifications: email, in-app alerts, calendar invites, schedule updates.

\subsubsection{Email Template}
Defines reusable templates with subject, body, placeholders. \texttt{renderTemplate()} fills them dynamically.

\subsection{Component UML}

\begin{figure}[htbp]
    \centering
    \includegraphics[width=1.0\textwidth]{../images/uml_component.png}
    \caption{Component Diagram of the TeachWise system.}
    \label{fig:component_diagram}
\end{figure}

\subsubsection{User Management Component}
Encapsulates user identity, authentication, and role-based access. Services:
\begin{itemize}
    \item User registration and login
    \item Role assignment
    \item Profile editing
    \item Parent-Student linking
\end{itemize}

\subsubsection{Class And Enrollment Management Component}
Manages lifecycle of classes, enrollments, and lessons. Interfaces with:
\begin{itemize}
    \item User Management
    \item Curriculum
    \item Scheduling
    \item Performance Evaluation
\end{itemize}

\subsubsection{Curriculum Component}
Contains \texttt{Course} and \texttt{Module} classes, managing instrcutor guides and any required resources.

\subsubsection{Scheduling Component}
Includes \texttt{Schedule}, \texttt{Scheduling Engine}, and \texttt{Matching Engine}. Automates schedules and matches students to instructors.

\subsubsection{Performance Evaluation Component}
Evaluates instructor performance using rubrics and ML-driven analysis. Integrates with \texttt{ZoomAPI} for recordings.

\subsubsection{Payment Component}
Manages compensation, payroll, bonuses, and approval workflows. Draws data from classes, lessons, and evaluations.

\subsubsection{Notification Service Component}
Centralizes all communication: email, calendar invites, in-app alerts. Integrates with all major subsystems.

\subsubsection{External Integration Component}
Houses third-party integrations: Zoom API, future payment gateways, analytics services, LMS tools. Supports modular upgrades without affecting internal components.

% Your report will contain ONE of the following 2 sections.

\section{Data Design}

This section presents the structure of our database that caters to persistent data storage in our project. The structure is shown as a normalized data model for relational databases. It clearly shows entities, attributes, relationships with their cardinalities, and primary and foreign keys. We have used DB designer to build our data model.

\begin{figure}[htbp]
    \centering
    \includegraphics[width=1.0\textwidth]{../images/teachwisecms_erd.png}
    \caption{Entity-Relationship Diagram of the database.}
    \label{fig:erd}
\end{figure}

\section{Technical Details}
\label{sec:technical}

This section presents the computational, algorithmic, and architectural components that form the core of TeachWise. Unlike conventional CRUD-based systems, TeachWise integrates intelligent scheduling, machine learning-based instructor evaluation, automated Zoom meeting generation, and multimodal data pipelines. For each subsystem, we describe the inputs, internal processes, and outputs, along with how these components integrate into the system-wide toolchain.

Only methods that are technically viable and supported by research evidence are included. These represent the approaches we intend to implement based on feasibility and relevance.

\subsection{Intelligent Scheduling Algorithm}

The platform uses an intelligent scheduling engine inspired by hybrid optimization techniques from recent literature \cite{lv2025, sahargahi2022, nguyen2021sho}. 

\subsubsection*{Inputs}
\begin{itemize}
    \item Instructor availability (days, time windows, months)
    \item Student timing preferences
    \item Instructor expertise (language, module, age group)
    \item Student proficiency and course requirements
    \item Instructor past performance ratings
    \item Zoom account availability
\end{itemize}

\subsubsection*{Method Employed}
We aim to implement a Hybrid Scheduling Model combining:

\begin{enumerate}
    \item \textbf{Improved Parallel Genetic Algorithm (GA)}  
    Generates multiple candidate schedules and evolves them using selection, mutation, and crossover \cite{sahargahi2022}.

    \item \textbf{Local Search Optimization}  
    Refines feasible schedules to minimize soft constraint violations \cite{sahargahi2022}.

    \item \textbf{Distance-to-Feasibility Fitness Function}  
    Ensures hard constraints (Zoom limits, overlapping classes, instructor conflicts) are always satisfied while optimizing soft constraints such as preferences, continuity, and compatibility \cite{sahargahi2022}.
\end{enumerate}

\subsubsection*{Output}
\begin{itemize}
    \item Conflict-free schedule with optimized soft constraints
    \item Automatic Zoom meeting assignment integrated into the schedule
\end{itemize}


\subsection{Student-Instructor Matching Model}

As outlined in the SRS, the system pairs students with instructors using a machine-learning-based matching model to enhance learning outcomes \cite{lv2025, almubarak2025}. 

\subsubsection*{Inputs}
\begin{itemize}
    \item Student profile: age, proficiency, enrolled modules
    \item Instructor features: expertise, experience, ratings history
    \item Overlaps in availability
    \item Historical pairing performance
\end{itemize}

\subsubsection*{Method Employed}
We will explore a hybrid evolutionary model integrating regression and classification methods. Prior research shows strong effectiveness for machine-learning-based teacher-student allocation \cite{nguyen2021sho}.

Features contributing to the predicted matching score include:
\begin{itemize}
    \item Language compatibility
    \item Curriculum overlap
    \item Instructor performance evaluation
    \item Time-zone and availability synchronization
\end{itemize}

\subsubsection*{Outputs}
\begin{itemize}
    \item Ranked instructor recommendations per student
    \item Matching scores fed into the GA scheduling model
\end{itemize}


\subsection{Zoom API Automation Pipeline}

The system automates meeting generation using the Zoom REST API \cite{zoomapi2025}, eliminating manual workflows.

\subsubsection*{Inputs}
\begin{itemize}
    \item Finalized class schedule
    \item Instructor Zoom credentials
    \item Meeting metadata (date, time, duration, class\_id)
\end{itemize}

\subsubsection*{Method Employed}
A secure backend service makes authenticated API calls to:
\begin{itemize}
    \item Programmatically create Zoom meetings
    \item Store meeting\_id, zoom\_link, and password
    \item Link meeting IDs to class entries
    \item Send notifications to parents, instructors, and students
\end{itemize}

\subsubsection*{Output}
\begin{itemize}
    \item Auto-generated Zoom links stored in the database
    \item Email notifications to all participants
    \item Meeting logs for reporting
\end{itemize}


\subsection{Instructor Performance Evaluation (ML/NLP Pipeline)}

The evaluation pipeline analyzes teaching quality using multimodal ML techniques, consistent with research demonstrating AI’s promise in classroom evaluation \cite{almubarak2025}.

\subsubsection*{Inputs}
\begin{itemize}
    \item Zoom audio recordings
    \item Whisper-generated transcripts
    \item Defined metrics: clarity, energy, patience, attention, fidelity, pacing
\end{itemize}

\subsubsection*{Method Employed}

The evaluation pipeline consists of several components:

\paragraph{A. Audio Feature Extraction}
\begin{itemize}
    \item OPENSmile for prosodic, spectral, and intensity features \cite{eyben2013, eyben2010}
    \item Librosa for tempo, spectral, and speech-related features \cite{eyben2013}
\end{itemize}

\paragraph{Supported metrics:}
Energy, engagement, patience (stress cues), pacing.

\paragraph{B. Speech-to-Text Processing}
\begin{itemize}
    \item Whisper for high-quality transcription of noisy audio
\end{itemize}

\paragraph{Supported metrics:}
Introduction clarity, curriculum fidelity.

\paragraph{C. NLP \& Semantic Analysis}
\begin{itemize}
    \item Contextual LLM-based clarity assessment
    \item Sentiment and tone analysis for patience \& positivity
    \item BERTopic for curriculum alignment \cite{grootendorst2022}
    \item Linguistic feature extraction (structure, pacing, Q/A pattern)
    \item Speaker diarization for teacher-student balance
\end{itemize}

\subsubsection*{Metric Implementation Summary}
\begin{enumerate}
    \item \textbf{Introduction Clarity:} BERT-based structure detection \cite{almubarak2025}
    \item \textbf{Energy \& Engagement:} Pitch, intensity, tempo variation via OPENSmile
    \item \textbf{Patience:} Acoustic stress/pausing cues
    \item \textbf{Student Attention:} Talk-time balance via diarization
    \item \textbf{Curriculum Fidelity:} BERTopic-based topic extraction
    \item \textbf{Lesson Pacing:} Speaking rate and topic progression
\end{enumerate}

\subsubsection*{Output}
\begin{itemize}
    \item Composite instructor performance score
    \item Metric-level breakdown stored in the \texttt{evaluations} table
    \item Insights powering scheduling priority and dashboards
\end{itemize}


\subsection{Data Flow and Integration Pipeline}

A modular architecture ensures seamless coordination between ML subsystems and the main operational backend.

\subsubsection*{Layers}
\begin{itemize}
    \item \textbf{Data Layer (PostgreSQL):} Stores instructors, students, schedules, evaluations, Zoom meetings.
    \item \textbf{Application Layer (Next.js/Rails):}  
    Handles scheduling engine API, Zoom API, and frontend integration.
    \item \textbf{ML Layer (Python/TensorFlow):}  
    Implements scheduling, evaluation, and matching.
    \item \textbf{Integration Layer:}  
    Email services, Zoom API gateway.
    \item \textbf{Presentation Layer (React/Next.js):}  
    Dashboards, schedules, analytics, class links.
\end{itemize}

\subsection{System Outputs}

The system produces the following outputs:

\begin{itemize}
    \item Conflict-free monthly class schedules
    \item Auto-generated Zoom links
    \item Student-Instructor match recommendations
    \item Instructor performance reports
    \item KPIs and admin dashboards
    \item Payment and attendance reports
\end{itemize}

These components collectively transform manual workflows into an intelligent, scalable, and automated platform.

\begin{figure}[htbp]
    \centering
    \includegraphics[width=1.0\textwidth]{../images/teachwisecms_erd.png}
    \caption{Entity-Relationship Diagram of the database.}
    \label{fig:erd}
\end{figure}